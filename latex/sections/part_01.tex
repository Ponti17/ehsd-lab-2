\documentclass[../main.tex]{subfiles}

\graphicspath{{\subfix{../imgs/}}}

\begin{document}

\section{Part 1 - Modelling and Simulations}

\subsection{Task 1.1}

Draw a complete equivalent circuit model of the buffer hip including all the signals of the drivers outputs, GDN and VCC supplies, the associated package parasitics for each of the paths and the PCB transmission lines including termination components.

\solution

To fully model the circuit for maximum accuracy we will have to create lumped element models for the supply, driver, receiver and probe. The transmission line will simply be modelled with a time delay and characteristic impedance.

\subsubsection{Supply Model}

The lumped element model for the supply is seen in Figure \ref{fig:supply} and component descriptions (as well as where we find their values) are summarized in Table \ref{tab:supply}.

\begin{figure}[h]
    \centering
    \fbox{\includegraphics[width=0.8\textwidth]{supply.png}}
    \caption{LTSpice supply model.}
    \label{fig:supply}
\end{figure}

\begin{table}[h]
    \centering
    \begin{tabular}{l|l r}
        \toprule[1pt]
        \textbf{Component} & \textbf{Description} & \textbf{Source}\\
        \midrule
        $C_1$       & Supply decoupling & (Board schematic) \\
        $C_2/C_4$   & 74ALVC244 VCC/GND pin capacitance & (IBIS) \\
        $L_1/L_2$   & 74ALVC244 VCC/GND pin inductance & (IBIS) \\
        $R_1/R_2$   & 74ALVC244 VCC/GND pin resistance & (IBIS) \\
        $C_3/C_5$   & 74ALVC244 input pin capacitance & (Datasheet) \\
        \bottomrule[1pt]
    \end{tabular}
    \caption{Supply model components descriptions.}
    \label{tab:supply}
\end{table}

\subsubsection{Driver/Receiver Model}

Next we model the driver, receiver and transmission line inbetween. Shown in Figure \ref{fig:driver} and summarized in Table \ref{tab:driver}. The buffer likely has a \textit{totem-pole} CMOS pair as the output stage. We model this with switches. The switches have an on resistance of $R_{on} = 1\si{\Omega}$ and an off resistance of $R_{off} = 10^6\si{\Omega}$.

\begin{figure}[h]
    \centering
    \fbox{\includegraphics[width=0.9\textwidth]{ltspice_model.png}}
    \caption{LTSpice model of the driver and receiver.}
    \label{fig:driver}
\end{figure}

\newpage

\begin{table}[h]
    \centering
    \begin{tabular}{l|l r}
        \toprule[1pt]
        \textbf{Component} & \textbf{Description} & \textbf{Source}\\
        \midrule
        $SW$    & CMOS totem pole output.       & \\
        $R_3$   & 74ALVC driver resistance      & \\
        $C_6$   & 74ALVC244 driver capacitance  & (IBIS) \\
        $R_5$   & Driver package resistance     & (IBIS) \\
        $L_3$   & Driver package inductance     & (IBIS) \\
        $C_7$   & Driver package capactiance    & (IBIS) \\
        $R_6$   & Board resistor                & (Board schematic)\\
        $T_1$   & PCB trace transmission line   & \\
        $C_8$   & Receiver package capacitance  & (IBIS) \\
        $R_7$   & Receiver package resistance   & (IBIS) \\
        $L_4$   & Receiver package inductance   & (IBIS) \\
        $C_9$   & Receiver input capactitance   & (IBIS) \\
        \bottomrule[1pt]
    \end{tabular}
    \caption{System model components descriptions.}
    \label{tab:driver}
\end{table}

\textbf{Note:} physical resistor $R_6$ is not present on all lines, in which case it will be set to 0.

\subsubsection{Probe Model}

As seen in the schematic the probe is attached to the 8th output of the buffer, which can either be set to low or high. We neglect the probes capacitances/inductances and simply model it as a $50\si{\Omega}$ resistor. The model is shown in Figure \ref{fig:probe}.

\begin{figure}[h]
    \centering
    \fbox{\includegraphics[width=0.9\textwidth]{probe.png}}
    \caption{LTSpice model of the probe.}
    \label{fig:probe}
\end{figure}

\subsubsection{Complete Model}

Finally we combine all the models into a single schematic. The complete model is shown in Figure \ref{fig:complete}. All 7 lines and the quiet probe line are modeleed. By configuring the voltage sources to the left we can switch 1, 3 or 7 of the lines. 

\newpage

\begin{figure}[h]
    \centering
    \fbox{\includegraphics[width=0.9\textwidth]{complete.png}}
    \caption{LTSpice model of system.}
    \label{fig:complete}
\end{figure}

\newpage

\subsection{Task 1.2}

Determine from the IBIS model for the driver/receiver IC the inductances of the signal, GND and VCC pins for the QFN and SOIC packages.

\solution

Most of the parameters are found from the IBIS file. The package parameters for both the QFN and SOIC packages are shown in Table \ref{tab:pkg-params}. The driver has an output capacitance of $C_{driver} = 6.03\si{pF}$ and the receiver has an input capacitance of $C_{receiver} = 4.21\si{pF}$. The time delay of the transmission is determined from the propagation speed listed in assignment 1, $\text{TD} = 0.1 \si{m} / 1.59 * 10^8\si{m/s} = 0.629 \si{ns}$. We assume the transmission line has $Z_0 = 50\si{\Omega}$.

\begin{table}[h]
    \centering
    \begin{tabular}{r|r r r r}
        \toprule[1pt]
        \textbf{Package} & \textbf{Pin} & $R_{pkg}$ [$\si{\Omega}$] & $C_{pkg}$ [$\si{pF}$] & $L_{pkg}$ [$\si{nH}$] \\
        \midrule
        SOIC  & 1Y1  & 0.034 & 0.731 & 4.714 \\
        SOIC  & 1Y2  & 0.036 & 0.533 & 4.130 \\
        SOIC  & 1Y3  & 0.035 & 0.596 & 4.414 \\
        SOIC  & 1Y4  & 0.045 & 0.868 & 5.561 \\
        SOIC  & 2Y1  & 0.040 & 0.865 & 5.518 \\
        SOIC  & 2Y2  & 0.034 & 0.583 & 4.284 \\
        SOIC  & 2Y3  & 0.036 & 0.533 & 4.130 \\
        SOIC  & 2Y4  & 0.034 & 0.732 & 4.714 \\
        \midrule
        SOIC  & 1A1  & 0.043 & 0.867 & 5.542 \\
        SOIC  & 1A2  & 0.034 & 0.591 & 4.377 \\
        SOIC  & 1A3  & 0.035 & 0.530 & 4.111 \\
        SOIC  & 1A4  & 0.034 & 0.726 & 4.634 \\
        SOIC  & 2A1  & 0.041 & 0.861 & 6.281 \\
        SOIC  & 2A2  & 0.033 & 0.742 & 4.778 \\
        SOIC  & 2A3  & 0.035 & 0.533 & 4.110 \\
        \midrule
        SOIC  & VCC  & 0.042 & 0.864 & 6.308 \\
        SOIC  & GND  & 0.043 & 0.866 & 6.349 \\
        \midrule
        QFN  & 1Y1  & 0.0973 & 0.138 & 1.40 \\
        QFN  & 1Y2  & 0.0933 & 0.128 & 1.33 \\
        QFN  & 1Y3  & 0.100  & 0.140 & 1.44 \\
        QFN  & 1Y4  & 0.118  & 0.159 & 1.73 \\
        QFN  & 2Y1  & 0.110  & 0.151 & 1.62 \\
        QFN  & 2Y2  & 0.0960 & 0.137 & 1.38 \\
        QFN  & 2Y3  & 0.0932 & 0.127 & 1.34 \\
        QFN  & 2Y4  & 0.102  & 0.144 & 1.48 \\
        \midrule
        QFN  & 1A1  & 0.111  & 0.151 & 1.62 \\
        QFN  & 1A2  & 0.0956 & 0.137 & 1.38 \\
        QFN  & 1A3  & 0.0932 & 0.127 & 1.34 \\
        QFN  & 1A4  & 0.102  & 0.144 & 1.48 \\
        QFN  & 2A1  & 0.0942 & 0.126 & 1.35 \\
        QFN  & 2A2  & 0.108  & 0.151 & 1.57 \\
        QFN  & 2A3  & 0.108  & 0.152 & 1.57 \\
        \midrule
        QFN  & VCC  & 0.0913 & 0.125 & 1.30 \\
        QFN  & GND  & 0.0458 & 0.154 & 1.00 \\
        \bottomrule[1pt]
    \end{tabular}
    \caption{Pin-wise package parameters for the 74ALVC244 buffer (SOIC/QFN).}
    \label{tab:pkg-params}
\end{table}

\newpage

\subsection{Task 1.3}

Simulate and document the voltages at the driver and receiving end on one of the switching outputs, as well as the quiet line.

\subsubsection{QFN Package}

During simulation we plot the near and far end of the transmission line i.e. right after the lumped element of the driver and right before the receiver. In addition we also plot the quiet line. The near and far end of the transmission line is shown in Figure \ref{fig:qfn_sim_near_far} and the quiet line is shown in Figure \ref{fig:qfn_sim_quiet}.
We see that the switching line on the driver end has some overshoot but stabilizes quickly compared to the voltage on the receiving end. 
The simulation also shows that the voltage overshoot on the receiving end is higher than the driver end, and oscillates around what would be the 
expected steady-state voltage, increasingly with the amount of enabled outputs. 

\vspace{10pt}
The quiet line shows similar behavior when increasing the amount of outputs, with increasing voltage spikes around switching times.

\begin{figure}[h]
    \centering
    \includegraphics[width=0.9\textwidth]{qfn_sim_near_far.pdf}
    \caption{LTSpice simulation of a switching line on the QFN package.}
    \label{fig:qfn_sim_near_far}
\end{figure}

\begin{figure}[h]
    \centering
    \includegraphics[width=0.9\textwidth]{qfn_sim_quiet.pdf}
    \caption{LTSpice simulation of the quiet line on the QFN package.}
    \label{fig:qfn_sim_quiet}
\end{figure}

\newpage

The ground and VCC bounce for the quiet line is shown in Table \ref{tab:qfn_bounce}. \textit{Were not really sure what voltage values you want for the switching line, so we will just summarize the maximums in a table.}

\begin{table}[h]
    \centering
    \begin{tabular}{l | r r}
        \toprule[1pt]
        \textbf{Num. of Outputs} & \textbf{GND [V]} & \textbf{VCC [V]} \\
        \midrule
        1 & 0.995 & 1.78 \\
        3 & 1.41  & 0.950 \\
        7 & 1.55  & 0.651 \\
        \bottomrule[1pt]
    \end{tabular}
    \caption{GND and VCC bounce (QFN)}
    \label{tab:qfn_bounce}
\end{table}

\begin{table}[h]
    \centering
    \begin{tabular}{r|c}
        \toprule[1pt]
        \textbf{Outputs} & \textbf{Switching Line[V]} \\
        \midrule
        1  & 3.59 \\
        3  & 4.13 \\
        7  & 4.78 \\
        \bottomrule[1pt]
    \end{tabular}
    \caption{Switching line maximum voltages (QFN).}
\end{table}

To determine if the values from the simulation are acceptable we will compare them to the noise margins of the 74ALVC244. As will be shown later the low noise margin $NM_L = 0.6\si{V}$, and the high noise margin $NM_H = 1.1\si{V}$. Meaning that the GND bounce should be no more than $0.6\si{V}$ and the VCC bounce should be no lower than $2.2\si{V}$.

\vspace{10pt}
Even when taking into the account the voltage loss due to the probe impedance of $50\si{\Omega}$, it looks \textit{pretty bad}. From the simulations we should expect a lot of errors.

\subsubsection{SOIC Package}

The SOIC package simulations shows different voltage curves over time. Here, the square wave is more visible,
and the voltage overshoots are more smoothed, since the reaction is slower. 
On the figures we can see that the oscillations becomes more elongated with the amount of enabled outputs, damping the higher
frequency components in the reaction. 

The critical voltages are somewhat higher than the QFN package, but we see a similar behavior.

\begin{figure}[h]
    \centering
    \includegraphics[width=0.85\textwidth]{soic_sim_near_far.pdf}
    \caption{LTSpice simulation of a switching line on the SOIC package.}
    \label{fig:soic_sim_near_far}
\end{figure}

\newpage

\begin{figure}[h]
    \centering
    \includegraphics[width=0.85\textwidth]{soic_sim_quiet.pdf}
    \caption{LTSpice simulation of the quiet line on the SOIC package.}
    \label{fig:soic_sim_quiet}
\end{figure}

\begin{table}[h]
    \centering
    \begin{tabular}{l | r r}
        \toprule[1pt]
        \textbf{Num. of Outputs} & \textbf{GND [V]} & \textbf{VCC [V]} \\
        \midrule
        1 & 1.15 & 1.58 \\
        3 & 1.46 & 1.21 \\
        7 & 1.55 & 1.26 \\
        \bottomrule[1pt]
    \end{tabular}
    \caption{GND and VCC bounce (SOIC)}
    \label{tab:soic_bounce}
\end{table}

\begin{table}[h]
    \centering
    \begin{tabular}{r|c}
        \toprule[1pt]
        \textbf{Outputs} & \textbf{Switching Line[V]} \\
        \midrule
        1  & 4.34 \\
        3  & 4.24 \\
        7  & 4.22 \\
        \bottomrule[1pt]
    \end{tabular}
    \caption{Switching line maximum voltages (SOIC).}
\end{table}

Again with the SOIC package we see that we are \textbf{far} above the noise margins both for GND and VCC bounce. \textit{Let's hope it will be better in the lab...}

\end{document}
